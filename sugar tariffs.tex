\documentclass[12pt]{article}

\addtolength{\textwidth}{.6in}
\addtolength{\oddsidemargin}{-.3in}
\setlength{\textheight}{8.4in}
\setlength{\topmargin}{-.25in}
\setlength{\headheight}{0.0in}
\renewcommand{\baselinestretch}{1.0}
\linespread{2}

\usepackage{cite}
\usepackage{times, verbatim,xcolor,bm}
\usepackage{amsbsy,amssymb, amsmath, amsthm}
\usepackage{hyperref}

\newtheorem{definition}{Definition}
\newtheorem{theorem}{Theorem}
\newtheorem{lemma}{Lemma}
\newtheorem{corollary}{Corollary}
\newtheorem{assumption}{Assumption}
\newtheorem{fact}{Fact}

\newcommand{\ve}{\varepsilon}
\newcommand{\ov}{\overline}
\newcommand{\un}{\underline}
\newcommand{\ta}{\theta}
\newcommand{\Ta}{\Theta}
\newcommand{\expect}{\mathbb{E}}

\begin{document}
\title{\vskip-0.6in Story and Timeline of Sugar Tariffs
}
\date{\today}
\maketitle



\href{http://www.american-historama.org/1881-1913-maturation-era/mckinley-tariff.htm}{McKinley Tariff: October 1, 1890} : "The McKinley Tariff had a dramatic effect on Hawaii.�Hawaii had long attracted the interest of American businessmen in the lucrative sugar trade. The United States federal government had provided generous terms to the sugar growers of Hawaii in the treaties of 1849 and 1875 and American businessmen had acquired substantial fortunes in the islands. The McKinley Tariff proved to a turning point in the relations between the United States and Hawaii. In 1890 the United States Congress approved the McKinley Tariff, which raised import rates on foreign sugar. This had an alarming effect on the sugar planters in Hawaii who, as a direct result of the McKinley Tariff, were being undersold in the American market. The McKinley Act removed the duty on all raw sugar coming into the United States, which deprived Hawaiian sugar producers of their privileged status. The powerful American sugar growers in Hawaii led by Lorrin A. Thurston, the leader of the "Hawaiian League", were agitating for the Annexation of Hawaii. They realized that if Hawaii were to be annexed by the United States of America, the tariff problem relating to the sugar would automatically disappear as Hawaii would no longer be a foreign country. "
\bigskip

Two Centuries of Tariffs (Richardson's book): only 1/10 of US demand for sugar was filled by the US. William McKinley put sugar on free list in 1890 and gave US producers 2 cent per pound bounty. Four years later, a portion of the old sugar duty was restored and the bounty was dropped. In 1897 the bounty was restored and the duty was kept. Was justified as "necessary for political reasons or to stimulate the growth of the domestic sugar supply" (23). Hawaii could no longer profit from sugar exports to the United States. In 1875, a trade agreement was negotiated that allowed Hawaii to export sugar to the US without tax. Hawaii reciprocated with reducing tariffs on certain imported US goods. "Between 1875 and 1890, US consumption of Hawaiian sugar increased over 1,400 percent" (24). The McKinley tariff eliminated reciprocity and Hawaii fell into economic crisis.
\bigskip

\href{https://evols.library.manoa.hawaii.edu/bitstream/10524/273/1/JL18005.pdf}{Alaska and Hawaii}�: Alaska was also seeking statehood and so Hawaii was deferred statehood until Alaska's was passed so Hawaii would have a more favorable outcome.�
\bigskip

\href{http://www.ushistory.org/us/44b.asp}{Annexation of Alaska and Hawaii} President McKinley annexed both islands during the Spanish war. Foreign competition dissipated for Hawaii.
\bigskip

In summary: The McKinley tariff lead to an economic crisis in Hawaii because of the heightened foreign competition after tariffs on sugar were cut. In order to restore the economy, annex a nation that was already sought after (especially for location because of the war), and import sugar on favorable terms, the US annexed Hawaii in 1900.
\bigskip

Richardson?s book "Two Centuries of Tariffs" gave me background on what events to look more deeply into. "A Charter for World Trade", "Traders and Diplomats", and "A Charter for World Trade" did not have anything on sugar or the McKinley Tariff.�
\bigskip



\href{https://scholarworks.montana.edu/xmlui/bitstream/handle/1/2554/WiltgenT1207.pdf}{AN ECONOMIC HISTORY OF THE UNITED STATES SUGAR PROGRAM}
\\
Prices spike when worldwide production is low. 
\\

1894, the sugar subsidy was replaced with a 40\% higher tariff than the 1890 tariff. 
\\

Sugar import quota system based on actual sugar imports between 1925 and 1933. A processing tax evened out the sugar value and the farm price and the Secretary was given power to change this tax. Quotas were used until 1959. 
\bigskip

\href{http://www.uvm.edu/~rsicotte/US\%20Sugar\%20Program.pdf}{1934 Sugar Quotas} "The sugar quota was adopted after the U.S. government determined that the long-standing policy using the tariff to protect the domestic industry was failing. A principal reason was that the tariff
was not raising the price of sugar because, by diminishing the imports of Cuban sugar, it
was causing severe decline in wages and costs on that island. In turn, Cuban sugar was
being offered at ever lower prices."
Industry List in Github: 1934, when voting the TAA for the first time at the initiative of the Republicans, in House and by Senator Overton, supported by a minority of Democrats. Concern the production stipulated in the AAA: sugar, coconut oil, sesame oil, laces, braids, wool and wheat. The goal of the amendments is to require the cost-production calculus for these particular products and to spare them from tariff cuts. The amendment is defeated even though it was supported by some Democrats.

\href{http://ufdcimages.uflib.ufl.edu/IR/00/00/09/54/00001/SC02200.pdf}{This is a really helpful site with timeline and description}
\\

Sugar Act of 1948-After WWII
\\

During WWII, much of the Sugar Act was suspended. "The post-war Sugar Act was similar to the 1934 and 1937 legislation in that it designated quotas for each supplier-area, and authorized marketing and acreage allotments for domestic and insular suppliers."
\bigskip

GATT: "In the last, the Uruguay Round (1986-94) had over 100 member nations engaged in a serious effort to eliminate, or at least reduce, agricultural subsidies, including sugar subsidies. 
In July 1988, as a result of drastic reductions in the total U.S. sugar import quota, Australia brought a formal complaint to the GATT, seeking the establishment of a panel to consider the charge that the U.S. sugar quota system was illegal under international trade law." This was accepted. So in 1990, the US imposed a tariff. "the United States implemented a two-tiered tariff scheme designed to satisfy the GATT's ruling on U.S. sugar quotas. The first step of the new plan consisted of sugar quotas and a low tariff of 0.625 cents per pound for foreign imports up to 1.725 million metric tons of raw sugar. The second step of the new program involved no sugar quotas for imports in excess of 1.725 million tons. However, a tariff of 16 cents per pound would be imposed on all sugar imports exceeding 1.725 million tons."



To do: after Smoot-Hawley tariff. See effects of Bretton-Woods, IMF, World Bank
\href{https://www.marketplace.org/2017/08/24/sustainability/trade-stories-globalization-and-backlash/what-was-one-worst-pieces-us-legislation}{Smoot-Hawley Tariff}

\end{document}
