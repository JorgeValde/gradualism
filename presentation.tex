%\documentclass{beamer} 
\documentclass[handout]{beamer} 
\usetheme{Ilmenau}
\usepackage{graphicx,verbatim,hyperref}
\usepackage{textpos}

\usecolortheme{beaver}
\useinnertheme{default}
\setbeamertemplate{itemize item}[triangle]
\setbeamertemplate{itemize subitem}[triangle]
\setbeamertemplate{itemize subsubitem}[circle]
\setbeamertemplate{enumerate items}[default]
\setbeamertemplate{blocks}[upper=block head,rounded]
\setbeamercolor{item}{fg=black}
\usefonttheme{serif} %should allow ccfonts to take effect

\usepackage{cite}
\usepackage{xcolor,bm}
\usepackage{amsbsy,amssymb, amsmath, amsthm}
\usepackage{booktabs}
%David miller's fonts
	\usepackage[T1]{fontenc}
	\usepackage[boldsans]{ccfonts}
	\usepackage[euler-hat-accent]{eulervm}

\newcommand{\al}{\alpha}
\newcommand{\expect}{\mathbb{E}}
\newcommand{\Bt}{B(\bm{\tau^a})}
\newcommand{\bta}{\bm{\tau^a}}
\newcommand{\btn}{\bm{\tau^{tw}}}
\newcommand{\ga}{\gamma}
\newcommand{\ve}{\varepsilon}
\newcommand{\ta}{\theta}

\newenvironment{changemargin}[2]{% 
  \begin{list}{}{% 
    \setlength{\topsep}{0pt}% 
    \setlength{\leftmargin}{#1}% 
    \setlength{\rightmargin}{#2}% 
    \setlength{\listparindent}{\parindent}% 
    \setlength{\itemindent}{\parindent}% 
    \setlength{\parsep}{\parskip}% 
  }% 
  \item[]}{\end{list}} 
	
	\let\Tiny=\tiny


\title[Temporary Trade Barriers: When Will They End?\hspace{2.35in}\insertframenumber/\inserttotalframenumber]{Temporary Trade Barriers: \\ When Will They End?}
\author[Kristy Buzard]{\texorpdfstring{Kristy Buzard\newline Syracuse University and The Wallis Institute  \newline\url{kbuzard@syr.edu}}{Kristy Buzard}}
\date{May 13, 2016}

\begin{document}
\maketitle
%\insertpresentationendpage removed b/c of appendix




\section{Overview}
\subsection{Preview}
\begin{frame}
\frametitle{The Questions}
\pause
\begin{enumerate}[<+->]
	\item How long do deviations from trade agreement tariffs last? 
  \item What are the determinants of renewals?
		\begin{itemize}
			\item They don't happen for \textit{no} reason at all
		\end{itemize}
\end{enumerate}
\end{frame}


\begin{frame}{Institutional Detail}
\begin{itemize}[<+->]
	\item Temporary trade barriers have potential to be renewed
		\begin{itemize}
			\item For Anti-dumping (AD) duties, initial five year term, renewal for five years
			\item For Safeguards, four then four; much rarer than AD
		\end{itemize}
	\item Authorized by WTO
	\item Very little in the agreements to guide the process
		\begin{itemize}
			\item WTO litigation gives gov'ts a LOT of latitude
		\end{itemize}
	\item In U.S., for AD:
		\begin{itemize}
			\item DOC initiates, determines whether dumping would continue
			\item ITC determines whether injury would recur / continue
		\end{itemize}
\end{itemize}
\end{frame}

 
\begin{frame}{Preview of Results}

\pause
The probability that an AD duty gets renewed
\pause
\begin{itemize}[<+->]
	\item Increases in lobbying effort
	\item Decreases in the MFN tariff
	\item Is invariant to the trading partner's tariff
	\item Increases in the profitability of the import-competing sector
	\item Increases in the strength of the lobby
	\item May be concave in the AD duty
\end{itemize}
\end{frame}

\begin{comment}
\subsection{}
\begin{frame}
\frametitle{Related Literature}
\small Protection for Sale: Grossman $\&$ Helpman (1994)
\begin{itemize}
  \item \footnotesize Empirics: Goldberg $\&$ Maggi (1999), Gawande $\&$ Bandyopadhyay (2000), Mitra, Thomakos, $\&$  Ulubasoglu (2002)
  \item \footnotesize Mitra, Thomakos, $\&$  Ulubasoglu (2006), Bombardini (2008)
	\item \footnotesize Trade Wars and Trade Talks: Grossman $\&$ Helpman (1995)
\end{itemize}

\vskip.05in
\small Political economy shocks
\begin{itemize}
	\item \footnotesize Feenstra $\&$ Lewis (1991), Bagwell $\&$ Staiger (2001, 2005)
\end{itemize}

\vskip.05in
\small Separated powers
\begin{itemize}
	\item \footnotesize Mansfield, Milner $\&$ Rosendorff (2000), Song (2008)
\end{itemize}

\vskip.05in
\small Political uncertainty
\begin{itemize}
	\item \footnotesize Milner $\&$ Rosendorff (1997), Le Breton $\&$ Zaporozhets (2007) %legislators are agents, symmetric case very hard
\end{itemize}
	
		%\item Autocracy vs. Democracy: Mitra, Thomakos and Ulubasoglu (2002), Aidt and Gassebner (2010)
\end{frame}
\end{comment}

\section{Model}
\subsection{Economic and Political Structure}
\begin{frame}{Timeline}
\pause

Taking trade agreement tariff and anti-dumping duties as given,
\pause
\begin{enumerate}[<+->]
	\item \textbf{Import-competing firms lobby DOC/ITC to renew AD duties}
	\item \textbf{Uncertainty is resolved}
	\item \textbf{DOC/ITC decide whether to renew duties}
	\item Private actors make production, consumption decisions
\end{enumerate}
\end{frame}


\begin{frame}{Economy}
\begin{itemize}
	\item Two countries: home and foreign (${}^*$)
	\item Separable in three goods: $X$ and $Y$ (traded) and numeraire
	\item Demand identical for both goods in both countries
	\item Supply: $Q_X^*(P_X) > Q_X(P_X)$ $\forall P_X$; symmetric for $Y$ 
		\begin{itemize}
			\item Home net importer of $X$, net exporter of $Y$
		\end{itemize}
\end{itemize}

\vskip.2in
\pause
Home levies $\tau$ on $X$, Foreign levies $\tau^*$ on $Y$
\pause
\begin{itemize}
	\item $P_X=P_X^W + \tau$ and $\pi_X(P_X)$ increasing in $\tau$
\end{itemize}

\pause
\vskip.2in
Non-tradable specific factors motivates political activity


\end{frame}


\begin{frame}{Political Structure}
In Home country (foreign is passive):
\pause
\begin{itemize}[<+->]
	\item Dept. of Commerce / Int'l Trade Commission
		\begin{itemize}[<+->]
			\item Can renew AD duties
			\item Susceptible to influence of lobbying, perhaps both direct and indirect
			\item Modeled in reduced form
		\end{itemize}
%\pause
	\item A Single Lobby
		\begin{itemize}
			\item Represents import-competing sector, $X$
		\end{itemize}	
\end{itemize}

\end{frame}


\subsection{The Players}
\begin{frame}
\frametitle{``Government''}
\pause
Renewal decision determined by complex process including DOC, ITC, pressure via other political bodies. Reduced form:
\pause
\[
  W_G = \mathit{CS}_X(\tau) + \ga(e,\ta) \pi_X(\tau) + \mathit{CS}_Y(\tau^*) + \pi_Y(\tau^*) + \mathit{TR}(\tau)
\]

\pause
\begin{itemize}[<+->]
	\item $\mathit{CS_i(\cdot)}$: consumer surplus
	\item $\pi_X(\tau)$: profits of import-competing industry
	\item $\pi_Y(\tau^*)$: profits of exporting industry
	\item $\mathit{TR}(\tau)$: tariff revenue
\end{itemize}
\end{frame}


\begin{frame}
\frametitle{``Government''}
\[
  W_G = \mathit{CS}_X(\tau) + \ga(e,\ta) \pi_X(\tau) + \mathit{CS}_Y(\tau^*) + \pi_Y(\tau^*) + \mathit{TR}(\tau)
\]

\pause
\begin{itemize}
	\item $\ga(e,\ta)$: weight on import-competing industry profits
		\begin{itemize}
			\pause
			\item $e$: lobbying effort
			\pause
			\item $\ta$: uncertain element in $G$'s preferences
		\end{itemize}
\end{itemize}

\pause
\vskip.1in
\begin{beamerboxesrounded}[upper=palette tertiary, shadow=true]{Assumption 1}
\begin{enumerate}
  \pause
	\item $\ga(e,\ta)$ is increasing and concave in $e$ for all $\ta \in \Theta$.
\end{enumerate}
\end{beamerboxesrounded}
\end{frame}


\begin{frame}
\frametitle{Lobby}
\vskip.2in
\pause
Lobby chooses effort to maximize:
\[
  \left\{1-\Pr\left[ \text{AD Renewal}\right]\right\} \ \pi(\tau^a) + \Pr\left[ \text{AD Renewal} \right]  \pi(\tau^{\mathit{ad}})  - e
\]

\vskip.1in
\pause
\begin{itemize}[<+->]
	\item $e$: Lobbying effort
	\item $\tau^a$: home import tariff under trade agreement
	\item $\tau^{\textit{ad}}$: home import tariff equivalent under anti-dumping duties
\end{itemize}

\end{frame}


\section{Uncertainty}
\subsection{}
\begin{frame}{Timeline}
\begin{enumerate}[<+->]
	\item \textbf{Import-competing firms lobby DOC/ITC to renew AD duties}
	\item {\color{gray} \textbf{Uncertainty is resolved}}
	\item \textbf{DOC/ITC decide whether to renew duties}
	\item {\color{gray} Private actors make production, consumption decisions}
\end{enumerate}
	
\end{frame}

\begin{frame}{Why uncertainty?}
\pause
\textbf{Government}
\pause
\begin{itemize}
	\item Renews AD duties if $G$ prefers $\tau^{\mathit{ad}}$ to $\tau^a$
\end{itemize}

\pause
\vskip.1in
\textbf{Lobby}
\pause
\begin{itemize}[<+->]
	\item Given $(\tau^a,\tau^{*a})$ and $\tau^{\mathit{ad}}$, lobby knows what $e$ is required to induce renewal
 	\item Lobby pays this $e$ if: \hskip.2in $\pi(\tau^{\mathit{ad}}) - e > \pi(\tau^a)$
\end{itemize}

\pause
\vskip.1in
\textbf{In Equilibrium}
\pause
\begin{itemize}[<+->]
	\item Firms only put forth effort when they know renewal will be granted
\end{itemize}

\end{frame}


\begin{frame}{What's this uncertainty about?}
\pause
Lobby must be able to trigger the original AD duty
\begin{itemize}[<+->]
	\item \textit{Can} think of non-adherence to MFN as eqm path dispute
	\item It could be justified, not motivated by politics/rent-seeking
	\item But still, firms lobby for duties they don't get
\end{itemize}

\pause
\vskip.1in
So what's the uncertainty about?
\pause
\begin{itemize}[<+->]
	\item Strength of evidence
	\item Probability foreign will retaliate or initiate dispute (indirect)
	\item $G$'s valuation of harm to industry, e.g. how politically important is industry?
\end{itemize}

\end{frame}


\section{Results}
\subsection{}
\begin{frame}{Timeline}
\begin{enumerate}[<+->]
	\item \textbf{Import-competing firms lobby DOC/ITC to renew AD duties}
	\item \textbf{Uncertainty is resolved}
	\item \textbf{DOC/ITC decide whether to renew duties}
	\item Private actors make production, consumption decisions
\end{enumerate}
\end{frame}


\begin{frame}{Government}
  $G$ renews AD duties if its utility is higher under AD duties than trade agreement tariff
	\pause
  \begin{itemize}
		\item Preferences are ex-ante uncertain through $\ta$
		\pause
		\item When does $G$ renew AD duties? \\
	\pause
  \vskip.1in
    $b(e,\tau^a,\tau^{\textit{ad}})$: probability $G$ prefers $\tau^{ad}$ to $\tau^a$ for a given effort level $e$
\end{itemize}

\pause
\vskip.25in
\begin{beamerboxesrounded}[upper=palette tertiary, shadow=true]{Lemma 1}
  The probability that $G$ renews AD duties is increasing and concave in lobbying effort $e$ $\left(\text{i.e. } \frac{\partial b}{\partial e} \geq 0, \ \frac{\partial^2 b}{\partial e^2} \leq 0  \right)$.
\end{beamerboxesrounded}

\end{frame}

\begin{frame}{Home's Trade Agreement Tariff}

\pause
\begin{beamerboxesrounded}[upper=palette tertiary, shadow=true]{Result 1}
  The total probability that $G$ renews AD duties is decreasing in the home trade agreement tariff $\tau^a$.
\end{beamerboxesrounded}

\vskip.25in
\pause
There's both a direct effect and an indirect effect through lobby's incentives, and both are negative:
\[
  \frac{\partial b}{\partial e} \frac{\partial e}{\partial \tau^a} + \frac{\partial b}{\partial \tau^a}
\]
\end{frame}

\begin{frame}{Foreign's Trade Agreement Tariff}

Assuming trading partner does not retaliate
\pause
\begin{itemize}
	\item No difference in foreign tariff under AD duty and $\tau^a$. So no effect on $G$'s incentives (either direct or indirect)
\end{itemize}


\vskip.25in
\pause
\begin{beamerboxesrounded}[upper=palette tertiary, shadow=true]{Result 2}
  The total probability that $G$ renews AD duties is unaffected by foreign's trade agreement tariff $\tau^a$.
\end{beamerboxesrounded}
\end{frame}


\begin{frame}{Profitability of Import-Competing Sector}

\pause
\textit{NOTE: this is not quite right, but some version of it will be} \\
Assume $\pi(\cdot)$ shifts up uniformly for all $\tau$.
\pause
\begin{itemize}[<+->]
	\item Convexity of profits $\Rightarrow$ $G$'s marginal benefit of providing protection goes up
	\item Convexity of profits $\Rightarrow$ return from lobbying increases
\end{itemize}


\vskip.25in
\pause
\begin{beamerboxesrounded}[upper=palette tertiary, shadow=true]{Result 3}
  The total probability that $G$ renews AD duties is increasing in the profitability of the import-competing sector.
\end{beamerboxesrounded}
\end{frame}


\begin{frame}{Exogenous Shifts in $\ga(e,\ta)$}

\pause
Assume $\ga(\cdot,\cdot)$ shifts up uniformly for all $(e,\ta)$ pairs.
\pause
\begin{itemize}[<+->]
	\item $G$ gives more weight to firms' benefit
	\item Lobbying incentives are unchanged
\end{itemize}

\vskip.25in
\pause
\begin{beamerboxesrounded}[upper=palette tertiary, shadow=true]{Result 3}
  The total probability that $G$ renews AD duties increases when the weighting function shifts up exogenously and uniformly.
\end{beamerboxesrounded}
\end{frame}


\begin{frame}{Protection from AD Duties}

\pause
When $\tau^{\textit{ad}}$ increases, two effects on $G$'s incentives:
\pause
\begin{itemize}[<+->]
	\item Social welfare decreases, pushes for decrease in renewal probability 
	\item (Over-weighted) import-competing profits increase, pushes for increase in renewal probability 
\end{itemize}

\vskip.2in
\pause
Indirect effect is of same sign as direct effect
\pause
\begin{itemize}[<+->]
	\item When $\tau^{\textit{ad}}$ (i.e. close to social optimum), second effect dominates $\Rightarrow$ increase in renewal probability
	\item Effect may be concave
\end{itemize}

\end{frame}


\section{Conclusion}
\subsection{}
\begin{frame}{Future Work}
\pause
\begin{itemize}[<+->]
	\item Comparative static on uncertainty measure
	\item Empirical work
	\item Extend model to include initial decision to grant protection.
		\begin{itemize}
			\item Explain variation in a lobby's incentives between original application of AD and renewal
			\item Lobby's choice between investing in productive vs. rent-seeking behavior while protected
		\end{itemize}
\end{itemize}
\end{frame}


\begin{frame}{Feedback, Comments from InsTED presentation}
\begin{itemize}[<+->]
	\item Could leverage 5-year renewal cycle vs. 4-year re-election cycle (Sarah Danzman, Indiana U)
	\item Look for Kara Reynold's ``Political Economy of Antidumping Reviews (Shushanik)
	\item Need to have first stage (Jee-Hyeong, Gary Lyn)
		\begin{itemize}
			\item Look at Park and Blonigen (Jee-Hyeong)
			\item Cost differential for firms to get initial duty vs. renewal?
			\item Some firms choose not to pursue renewal when DOC contacts them 30-days before (Sasha)
		\end{itemize}
	\item Lobbying effort may be endogenous: if prob. of success is lower, maybe it's not worth providing effort (Mostafa)
	\item Uncertainty could be, in part, about how mobile labor is (Kishore)
\end{itemize}

\end{frame}



\end{document}