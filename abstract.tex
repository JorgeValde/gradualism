\documentclass[12pt]{article}

\addtolength{\textwidth}{.6in}
\addtolength{\oddsidemargin}{-.3in}
\setlength{\textheight}{8.4in}
\setlength{\topmargin}{-.25in}
\setlength{\headheight}{0.0in}
\renewcommand{\baselinestretch}{1.0}
\linespread{1.15}

\usepackage{cite}
\usepackage{times, verbatim,xcolor,bm}
\usepackage{amsbsy,amssymb, amsmath, amsthm}
\usepackage{hyperref}

\newtheorem{definition}{Definition}
\newtheorem{theorem}{Theorem}
\newtheorem{lemma}{Lemma}
\newtheorem{corollary}{Corollary}
\newtheorem{assumption}{Assumption}
\newtheorem{fact}{Fact}

\newcommand{\ve}{\varepsilon}
\newcommand{\ov}{\overline}
\newcommand{\un}{\underline}
\newcommand{\ta}{\theta}
\newcommand{\Ta}{\Theta}
\newcommand{\expect}{\mathbb{E}}

\begin{document}
\title{\vskip-0.6in Explaining Gradualism in Trade Liberalization: \\A Political Economy Approach}
%\author{Kristy Buzard}
\date{\today}
\maketitle

A notable feature of many international trading relationships is the gradual way in which barriers to trade have been dismantled in the post-war period (Bhagwati, 1988). Most notably, large-scale tariff reductions within the framework of the GATT/WTO have come as a result of a series of nine rounds of negotiations that began in 1947. From a theoretical point of view, it is not clear why this process should either proceed in stages or be inherently time consuming. But if one begins from the assumption that free trade is efficient, it makes sense to look for mechanisms related to whatever frictions have led to trade being restricted in the first place.

I argue that one important reason that inefficient tariffs are maintained is the exertion of political power by inefficient import-competing industries. When an industry encounters a shock to its support such as a key politician losing an election or an important committee position, its ability to maintain its current level of protection is reduced. The resulting loss of protection and therefore rents can lead to erosion of future political power and accompanying protection. I make this argument using a dynamic model of political economy. I plan to provide supportive evidence from peril point investigations in the early rounds of the GATT negotiations.

\vskip.2in
\large\textbf{Explanations for Gradualism} \\
\normalsize Various approaches have been used in the literature to model gradualism in trade liberalization. In Devereaux (1997), increasing benefits of integration to consumers gradually increase the costs of trade wars and lead to free trade over time. Similarly, Chisik's (2003) assumption that capacity accumulation in the export sector is partially irreversible leads to a gradually increasing dependence of export producers on trade and therefore increases countries' incentives to lower tariffs in successive negotiating rounds.

Several other papers have focused on mechanisms involving the import-competing sector. Staiger's (1995) model focuses on reductions in the size of the import-competing sector, with gradualism arising from the presence of workers with specialized skills that allow them to earn rents in the import-competing industry. Each successive round of trade liberalization displaces a small percentage of these workers, who then lose their rent-earning skill with some exogenous probability. When this occurs, they have no rents to protect in subsequent liberalization rounds and further tariff cuts can occur. In the context of unilateral trade opening, Mehlum (1998) demonstrates that gradual tariff reductions can improve welfare in the presence of a minimum wage, whereas Mussa (1986) shows similar results by assuming the presence of adjustment costs that are convex in the number of workers leaving the import-competing sector. Furusawa and Lai (1999) show a similar result in the context of an infinitely repeated tariff setting game between governments, and Zissimos (2007) demonstrates that the GATT requirement that punishments be limited to the `withdrawal of equivalent concessions' generates gradualism.

\vskip.2in
\large\textbf{Economic and Political Organization} \\
\normalsize 
I seek an explanation of the observed gradual reductions in barriers to trade that is fundamentally one of political economy, that is, a story that does not hinge on the specific nature of trade. The hope is that \textit{if} such an explanation exists, important lessons could be learned from the experience of trade liberalization under the GATT that could be applied to other issue areas in which international cooperation has been more circumscribed.

Even in the context of Staiger's (1995) model, gradualism can be halted if the factors of production in the import-competing sector are able to organize. In his formulation, each time the tariff is lowered, a few of the workers who were previously able to earn rents there leave the import-competing sector and this leads the government to further lower the tariff in the next time period because for any given level of protection, there is less to be gained while the costs of distortions remain the same. Under the right set of parameter values, this eventually leads to an elimination of tariffs and therefore rents for all workers.

If the workers could organize, those workers who would have remained in the import-competing industry would find it in their interests to pay the workers who would otherwise leave enough so that they would find it attractive to stay. This would curtail the process of tariff reductions and preserve the rents that all workers earn into the indefinite future. As soon as we assume that economic power is somehow organized or concentrated, we should also consider that this might translate into political power that could be wielded to influence the policy-making process in favor of those who hold it.

To model the influences of organized special interests, I follow the ``political support'' approach that was crystalized in Grossman and Helpman (1994). Incumbent politicians maximize a welfare function that depends on a sum of consumer surplus, producer surplus and tariff revenue, where the weight on the producer surplus in an industry is a function of the effort that is exerted by the industry's lobby. 

In the simplest version of the model, I focus on a small, open economy with one import-competing sector and one export sector. I assume there is only one lobby, which represents the import-competing sector. I assume output does not depreciate and that there are no capital markets. The firm's objective function is to maximize profits net of lobbying effort over the infinite future horizon. With no borrowing, there is a period-by-period non-negativity constraint on firm's wealth (the state variable). The firm can choose to contribute more than its profits from a given period, but only up to the amount it has saved. 

An important new feature here is that the import-competing industry can make strategic decisions in the current period in order to impact its future payoffs; the import-competing industry is now ``organized'' and can therefore make tradeoffs at the margin that the unorganized workers in Staiger (1995) could not make.

\vskip.2in
\large\textbf{Intuition for Politically-driven Gradualism} \\
\normalsize 
Buzard (2016) shows that in a simple repeated game where lobbying effort simply determines who is the median policy-maker, the trade agreement tariff decreases if the political weighting function shifts downward. That is, if it becomes more difficult to translate lobbying effort into political support, the lobby will receive lower levels of tariff protection in equilibrium. This means that if there is an unanticipated negative shock to the import-competing industry's political support---imagine that a key legislator is not re-elected, dies, or loses an important committee assignment---the trade agreement tariff falls. If changes in the political support function are exogenous, this mechanism would lead to the gradual reduction of tariffs only if there is strong serial correlation in the shocks.

Changes in the political support function---that is, how lobbying effort is translated into the weight the median policy-maker places on the import-competing industry's profits---are unlikely to be unaffected by the level of protection the lobby receives. In particular, we can think of the industry as making a joint decision about investments in productive activities and efforts to get supportive politicians elected/re-elected. This decision will be influenced by the prevailing domestic price, and thus, by the level of protection agreed in the immediately-preceding round of trade negotiations.

The relationship between investments in productive capacity and political support determine when gradualism can arise through this mechanism. If there is sufficient substitutability between these investments, the reduction in tariffs could drive increased political support and ward off further declines in protection. However, if these investments are complements, lower tariffs will lead to reduced production as well as smaller investments in future political support and a further reduction in future tariffs.
				

\newpage
\noindent\large\textbf{References}\\

\noindent\normalsize Bhagwati, J., 1988. Protectionism. MIT Press, Cambridge, MA. \\

\noindent Buzard, Kristy (2016), ``Self-enforcing Trade Agreements and Lobbying,'' Available at \url{https://kbuzard.expressions.syr.edu/wp-content/uploads/Self-enforcing-Trade_Agreements.pdf}. \\

\noindent Chisik, R., 2003. ``Gradualism in free trade agreements: a theoretical justification.'' Journal of International Economics, 59, 367-397. \\

\noindent Devereux, M., 1997. ``Growth, specialization, and trade liberalization.'' International Economic Review
38, 565-585. \\

\noindent Furusawa, T., Lai, E., 1999. ``Adjustment costs and gradual trade liberalization.'' Journal of International
Economics 49, 333-361. \\

\noindent Grossman, G. M., and E. Helpman, 1994. ``Protection for Sale.'' American Economic Review, 84, 833-850. \\

\noindent Maggi, G. and A. Rodr\'{i}guez-Clare, 2007. ``A Political-Economy Theory of Trade Agreements.'' American Economic Review, 97, 1374-1406. \\

\noindent Mehlum, H., 1998. ``Why gradualism?'' The Journal of International Trade and Economic Development 7, 279-297. \\

\noindent Mussa, M., 1986. ``The adjustment process and the timing of trade liberalization.'' In: Choksi, A., Papageorgiou, D. (Eds.), Economic Liberalization in Developing Countries. Basil Blackwell, Oxford. \\

\noindent Staiger, R., 1995. ``A theory of gradual trade liberalization.'' In: Levinsohn, J., Deardorff, A., Stern, R.
(Eds.), New Directions in Trade Theory. University of Michigan Press, Ann Arbor, MI, pp. 249-284. \\

\noindent Zissimos, B, 1997. ``The GATT and gradualism.'' The Journal of International Economics, 71, 410-433.

\end{document}