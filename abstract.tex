\documentclass[12pt]{article}

\addtolength{\textwidth}{.6in}
\addtolength{\oddsidemargin}{-.3in}
\setlength{\textheight}{8.4in}
\setlength{\topmargin}{-.25in}
\setlength{\headheight}{0.0in}
\renewcommand{\baselinestretch}{1.0}
\linespread{1.15}

\usepackage{cite}
\usepackage{times, verbatim,xcolor,bm}
\usepackage{amsbsy,amssymb, amsmath, amsthm}

\newtheorem{definition}{Definition}
\newtheorem{theorem}{Theorem}
\newtheorem{lemma}{Lemma}
\newtheorem{corollary}{Corollary}
\newtheorem{assumption}{Assumption}
\newtheorem{fact}{Fact}

\newcommand{\ve}{\varepsilon}
\newcommand{\ov}{\overline}
\newcommand{\un}{\underline}
\newcommand{\ta}{\theta}
\newcommand{\Ta}{\Theta}
\newcommand{\expect}{\mathbb{E}}

\begin{document}
\title{\vskip-0.6in Explaining Gradualism in Trade Liberalization: \\A Political Economy Approach}
\author{Kristy Buzard}
\date{November 14, 2009}
\maketitle

For 2017 CEA:
A notable feature of many international trading relationships is the gradual way in which barriers to trade have been dismantled in the post-war period. Most notably, large-scale tariff reductions within the framework of the GATT/WTO have come as a result of a series of nine rounds of negotiations that began in 1947. From a theoretical point of view, it is not clear why this process should either proceed in stages or be inherently time consuming. But if one begins from the assumption that free trade is efficient, it makes sense to look for mechanisms related to whatever frictions have led to trade being restricted in the first place. I argue that one important reason that inefficient tariffs are maintained is the exertion of political power by inefficient import-competing industries. When an industry encounters a shock to its support such as a key politician losing an election or an important committee position, its ability to maintain its current level of protection is reduced. The resulting loss of protection and therefore rents can lead to erosion of future political power and accompanying protection. I make this argument using a dynamic model of political economy and provide supportive evidence from peril point investigations in the early rounds of the GATT negotiations.

A notable feature of many international trading relationships is the gradual way in which barriers to trade have been dismantled in the post-war period (Bhagwati, 1988). Most notably, large-scale tariff reductions within the framework of the GATT/WTO have come as a result of a series of nine rounds of negotiations that began in 1947 and continue to the present.

From a theoretical point of view, it is not clear why this process should either proceed in stages or be inherently time consuming. But if one begins from the assumption that free trade is efficient, it makes sense to look for mechanisms related to whatever frictions have led to trade being restricted in the first place.

I will argue that one important reason why inefficient tariffs are maintained is the exertion of political power by inefficient import-competing industries. I will then present a dynamic model of political economy that is designed to answer the following question: can political economy help explain the observed gradual reductions in barriers to trade?


\vskip.4in
\large\textbf{Explanations for Gradualism} \\
\normalsize Various approaches have been used in the literature to model gradualism in trade liberalization. In Devereaux (1997), increasing benefits of integration to consumers gradually increase the costs of trade wars and lead to free trade over time. Similarly, Chisik's (2003) assumption that capacity accumulation in the export sector is partially irreversible leads to a gradually increasing dependence of export producers on trade and therefore increases countries' incentives to lower tariffs in successive negotiating rounds.

Several other papers have focused on mechanisms involving the import-competing sector. In the context of unilateral trade opening, Mehlum (1998) demonstrated that gradual tariff reductions can improve welfare in the presence of a minimum wage, whereas Mussa (1986) showed similar results by assuming the presence of adjustment costs that are convex in the number of workers leaving the import-competing sector.

Furusawa and Lai (1999) showed a similar result in the context of an infinitely repeated tariff setting game between governments. Staiger's (1995) model is similar in that it focuses on reductions in the size of the import-competing sector, but the mechanism is quite different. Here, gradualism arises from the presence of workers with specialized skills that allow them to earn rents in the import-competing industry. Each successive round of trade liberalization displaces a small percentage of these workers, who then lose their rent-earning skill with some exogenous probability. When this occurs, they have no rents to protect in subsequent liberalization rounds and further tariff cuts can occur.

\vskip.4in
\large\textbf{Economic and Political Organization} \\
\normalsize The forces underlying Staiger's (1995) model of gradualism can be halted if the factors of production in the import-competing sector are able to organize. In his formulation, each time the tariff is lowered, a few of the workers who were previously able to earn rents there leave the import-competing sector and this leads the government to further lower the tariff in the next time period because for any given level of protection, there is less to be gained while the costs of distortions remain the same. Under the right set of parameter values, this eventually leads to an elimination of tariffs and therefore rents for all workers.

If the workers could organize themselves, those workers who would have remained in the import-competing industry would find it in their interests to pay the workers who would otherwise leave enough so that they would find it attractive to stay. This would curtail the process of tariff reductions and preserve the rents that all workers earn into the indefinite future.

Given that we regularly observe economic groups of many types organizing themselves for various purposes, it seems worthwhile to inject this aspect of realism into a model of tariff setting such as those discussed above. As soon as we assume that economic power is somehow organized or concentrated, we should also consider that this might translate into political power that could be wielded to influence the policy-making process in favor of those who hold it.

To model the influences of the political power of the import-competing sector I will follow Grossman and Helpman (1994). In this ``political support'' approach, lobbies bid for trade protection with their campaign contributions while incumbent politicians maximize a welfare function that depends on a weighted sum of total contributions and the welfare of voters. Thus we can think of the policy maker as maximizing his probability of being re-elected, which is increasing in the financial contributions he receives for providing tariff protection for the import-competing industry but decreasing in the social costs those tariffs impose on voters.

In their model, Grossman and Helpman take the set of industries that are politically organized as exogenous and then imagine that each lobby presents the government with a contribution schedule that maps tariffs into a campaign contribution level. Taking these contribution schedules as given, the government sets a policy vector and collects the contributions associated with it. A significant portion of the analysis is devoted to understanding how special interest groups compete and which will reap the most rewards from their participation in the political process. In order to focus on how the influence of a lobby may change endogenously over time, I will simplify this significantly by having only one lobby. I will make this lobby's contribution schedule and the the government's welfare function precise in the next section.


\vskip.4in
\large\textbf{Model} \\
\normalsize In this model, I will explore how two large, symmetric countries might interact when setting trade policy. This will be a fully dynamic, infinite-horizon model in discrete time. There are two countries, Home ($H$) and Foreign ($F$); one factor of production, labor ($l$); a non-traded good ($N$) and two manufacturing goods ($X$ and $M$). $N$ is produced one-for-one from labor, while the production functions for the manufacturing goods in each country $i$ are $X_i = A^i l$ and $M_i = B^i l$ where $A^H > A^F$ and $B^H < B^F$ so that under free trade, Home will export $X$ and import $M$.

[I need to talk to some trade people to figure out whether it's best to specify consumer preferences directly or indirectly by way of demand functions, as a lot of this literature seems to do; once determined, these specifications will go here]

I follow Staiger and assume that the labor in the export sector is specific to that sector and therefore no movements into or out of this sector are possible in either country. This is done to focus the analysis on the dynamics in the import-competing sector. I also assume that the export sector in each country is perfectly competitive and therefore unable to lobby the government for protection.

In the import-competing sector, I assume that all production is carried out by a single firm that can also choose to lobby the government. I also assume that output does not depreciate and that there are no capital markets so that the owner can save output but cannot borrow against future earnings. Then the firm's objective function is to maximize profits net of campaign contributions, $C_t$:
\[
  \max_{l_t,\tau_t} \sum_{t=0}^\infty \left\{B^H l_t(p_M^F + \tau^H) - w_t l_t - C_t(\tau_t) \right\} \hskip.2in \text{s.t.} \hskip.2in W_t \geq 0
\]
where $W_t$ is total wealth. The idea behind the non-negativity constraint on wealth is that the firm can choose to contribute more than its profits from a given period, but only up to the amount it has saved. Importantly, in the first period, it can only overspend profits by the amount of its initial wealth. The firms' wealth (assets? value? what's the right word here) will serve as the state variable of this problem, and I hope that $W_0$, along with $\tau_0^H$ and $\tau_0^F$ (the initial tariffs on the import good set by home and foreign respectively), will provide interesting comparative statics.

Note that the domestic monopolist cannot set prices or quantity as a monopolist because it is constrained by the perfectly inelastic supply of goods from the foreign country at effective price $p_M^F + \tau^H$. The important new feature here is that the monopolist can make strategic decisions in the current period in order to impact its future payoffs; the import-competing industry is now ``organized'' and can therefore make tradeoffs at the margin that the unorganized workers in Staiger (1995) could not make.

We need to specify the government's welfare function and the precise timing of actions in order to determine how the firm will make these decisions. If the politician in power were completely selfless and sought only to maximize the welfare of voters, lobbying could not have an impact on his decision.\footnote{This might not hold if the government does not have full information and the lobby can communicate information through its contribution schedule, but we assume full information.}  Instead, as discussed above, I follow Grossman and Helpman (1994) in assuming that an incumbent politician maximizes his political support with the goal of being re-elected. Although the election itself is not explicitly modeled, our politicians trade off financial contributions from the lobby against the loss of support from voters that results from implementing socially-suboptimal policies. Grossman and Helpman show that one can model political conflict such as that between the import-competing firm owner and the rest of society with a weighted social welfare function such as the following:
  \[
		G(\tau) = C(\tau) + aW(\tau)
	\]
where $C$ is the lobby's contribution function, $\tau$ is the tariff level, and $W$ is aggregate welfare. Once I have specified preferences / demand, I should be able to easily formulate $W(\tau)$.

\vskip.2in
I model the timing within each period, which we might think of as an electoral cycle, as follows:
		\begin{enumerate}
			\item Lobby formulates and delivers contribution schedule $C_t(\tau_t)$
			\item Government chooses $\tau_t$
			\item Lobby pays contribution, government enforces tariff
			\item Production takes place, workers are paid
			\item Tariff revenue is distributed and consumption takes place
			\item Election occurs (not explicitly modeled)
		\end{enumerate}

\vskip.4in
\large\textbf{Game Between Lobby and Government}\\
\normalsize The strategic interaction between the lobby and the government occurs in at least two stages. One might imagine the lobby presenting the government with a correspondence listing contribution levels for all possible tariffs; then the government would run each pair though its objective function and choose the one that provides it with the highest level of utility. This is not only computationally intense, but also not terribly realistic. As the lobby must be a willing participant in the process, in some sense it has all the bargaining power. Then modeling the lobby's offer as ``take-it-or-leave-it'' seems reasonable. I will proceed in this fashion since it both seems realistic and greatly simplifies the analysis.

In order to maximize its own welfare, the lobby should not offer any higher level of $C_t$ for any given $\tau_t>0$ than that which makes the government just indifferent between choosing that positive $\tau_t$ and the zero tariff that would leave the economy undistorted and total welfare at its maximum. That is, the lobby formulates $C_t(\tau_t)$ using the government indifference condition\footnote{Note that under this formulation, it doesn't matter whether we think of the lobby as choosing $C_t$ or $\tau_t$ because they're linked together one-to-one by the indifference condition.}
  \[
	  C(\tau) + a W(\tau) = a W(0)
	\]
Rearranging slightly, we have
  \[
	  C(\tau) = a \left[ W(0) - W(\tau) \right]
	\]
This condition has a very intuitive interpretation, as pointed out by Grossman and Helpman (1994): the lobby must contribute an amount proportional to the amount of welfare loss its desired tariff creates, where the factor of proportionality is the $a$ parameter in the government's welfare function.

Once the $C(\tau)$ correspondence has been calculated, the lobby simply chooses the $C_t, \tau_t$ pair that maximizes its objective function and makes that its take-it-or-leave-it offer. Then the government will, by design, be indifferent between receiving the offered $C_t$ and making the lobby's desired $\tau_t$ policy and receiving no campaign contributions and setting a zero tariff; as is standard in these games, we assume the government does as the party with bargaining power wants (we could insert an $\varepsilon$ margin to make the government strictly prefer the lobby's offer).

				
\vskip.4in
\large\textbf{Game Between Governments} \\
\normalsize Once we have determined how each government and its respective lobby interact, we imbed these relationships within an infinitely repeated tariff setting game between the governments of the two production economies. The assumptions one makes about the enforceability of international agreements have important impacts on the outcomes, so I plan to explore some of the most common. Most of the literature cited above assumes that international agreements must be perfectly self-enforcing, but a recent paper by Maggi and Rodr\'{i}guez-Clare (2007) assume that they are perfectly enforceable. 

I am not yet familiar with how the latter analysis proceeds, but the former is a fairly straightforward repeated prisoner's dilemma style analysis. One calculates the tariff level that achieves the highest welfare level and then attempts to support it with grim-trigger style Nash reversion punishments. Gradualism is a possibility if the cooperative tariff level is one in which the rents accruing to the lobby are reduced sufficiently to place campaign contributions and therefore tariffs on a downward path. 

My immediate plan is to follow the broad outlines of the analysis in both Staiger (1995) and Maggi and Rodr\'{i}guez-Clare (2007) to see if either produce interesting dynamics concerning the path of tariffs.


\newpage
Problem with Maggi and Rodr\'{i}guez-Clare: their gradualism comes from the fact that capital owners are only temporarily stuck in the import-competing industry; they want protection in the near future while they're getting themselves unstuck, but don't care in the long run when they can switch to other sectors. This doesn't mesh with my idea of people who don't want to lose their political power at all and need to be killed off. \\

They say that with the ability to commit, there would be no gradualism in all these models like Staiger's. But once I put in political economy motive, I don't think that's the case. Gov't does what's best for lobby instead of country as a whole...
\begin{itemize}
	\item ``in these models, trade liberalization would occur at once if agreements were perfectly enforceable, or if players were sufficiently patient. In our model, on the other hand, gradualism emerges even though agreements are perfectly enforceable; as we remarked above, gradualism in our model is a consequence of the interaction
between frictions in capital mobility and lobbying by capital owners.''
\end{itemize}

Under perfect capital mobility, there can't be any rents in the ex-post stage, so lobby isn't willing to pay to compensate for long-run distortions associated with protection. Optimal agreement is free trade.
\begin{itemize}
	\item If capital is completely fixed, there is no need for domestic commitment.
\end{itemize}

Cites from Staiger (1994)
\begin{itemize}
	\item Endogenous free trade agreements and the multilateral trading system \\
E Ornelas - Journal of International Economics, 2005 - Elsevier
\end{itemize}

\newpage
\noindent\large\textbf{References}\\

\noindent\normalsize Bhagwati, J., 1988. Protectionism. MIT Press, Cambridge, MA. \\

\noindent Chisik, R., 2003. ``Gradualism in free trade agreements: a theoretical justification.'' Journal of International Economics, 59, 367-397. \\

\noindent Devereux, M., 1997. ``Growth, specialization, and trade liberalization.'' International Economic Review
38, 565-585. \\

\noindent Furusawa, T., Lai, E., 1999. ``Adjustment costs and gradual trade liberalization.'' Journal of International
Economics 49, 333-361. \\

\noindent Grossman, G. M., and E. Helpman, 1994. ``Protection for Sale.'' American Economic Review, 84, 833-850. \\

\noindent Maggi, G. and A. Rodr\'{i}guez-Clare, 2007. ``A Political-Economy Theory of Trade Agreements.'' American Economic Review, 97, 1374-1406. \\

\noindent Mehlum, H., 1998. ``Why gradualism?'' The Journal of International Trade and Economic Development 7, 279-297. \\

\noindent Mussa, M., 1986. ``The adjustment process and the timing of trade liberalization.'' In: Choksi, A., Papageorgiou, D. (Eds.), Economic Liberalization in Developing Countries. Basil Blackwell, Oxford. \\

\noindent Staiger, R., 1995. ``A theory of gradual trade liberalization.'' In: Levinsohn, J., Deardorff, A., Stern, R.
(Eds.), New Directions in Trade Theory. University of Michigan Press, Ann Arbor, MI, pp. 249-284. \\

\end{document}